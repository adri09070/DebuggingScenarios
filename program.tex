\documentclass[manuscript, language=english, language=french, nonacm]{acmart}

\begin{document}

\title{Le Debugger Pharo : utilisation de base et avancée\\
Centre Inria de l'Université de Lille}
\author{Steven Costiou}
\email{steven.costiou@inria.fr}
\author{Valentin Bourcier}
\email{valentin.bourcier@inria.fr}
\author{Adrien Vanègue}
\email{adrien.vanegue@inria.fr}




\begin{abstract}
    
Cette formation sur le debugger Pharo vise à fournir une compréhension approfondie de l'utilisation et des fonctionnalités de base et avancées du debugger.
La formation dure 6 heures.
Le premier objectif est de couvrir de manière solide les outils du quotidien, dans leur utilisation et leur configuration.
Le second objectif de la formation est de donner les pleins pouvoirs sur l'environnement de debugging.
D'une part, il s'agit de se sensibiliser aux outils spécifiques à certains scénarios ainsi qu'aux outils avancés (comme le time-traveling debugger).
D'autre part, la formation introduit les capacités d'extension du debugger pour adapter les outils à des domaines ou des problèmes spécifiques.
Finalement, un temps d'échange permettra un retour sur la formation elle même et sur les outils présentés.



\end{abstract}

\maketitle

%\newpage
\section{Objectifs de la formation} 

\section{Le debugger : outils du quotidien (45 minutes)}
 
\subsection{Présentation de la structure du debugger}

\begin{enumerate}
    \item La pile d'appel et ses options
    \item Les commandes et leurs extensions       
    \item L'éditeur de code
    \item L'inspecteur
\end{enumerate}

\subsection{Outils de base}

\begin{enumerate}
    \item Les points d'arrêt
    \item Les \textit{Debug Points} (Pharo12)        
    \item Génération d'assertions (Pharo 12)
\end{enumerate}

\subsection{Configuration du debugger}

\begin{enumerate}
    \item Options générales
    \item Debugger le debugger
    \item Les extensions
    \item Choisir un debugger
\end{enumerate}

\section{Object-centric debugging (2 heures)}

\begin{enumerate}
    \item Présentation de la méthode
    \item Presentation de l'outil : les points d'arrêt centrés objet
    \item Expérience empirique (participation facultative)
\end{enumerate}

\section*{Pause déjeuner}

\newpage
\section{Outils avancés I : Chest (30 minutes)}

\textit{Stockez et accédez à vos objets à volonté !}

\begin{enumerate}
    \item Présentation de l'outil
    \item Illustration et démonstration sur des scénarios types
\end{enumerate}

\section{Commandes avancées (1h)}

\begin{enumerate}
    \item Présentation des commandes avec exemples
    \item Création de vos propres commandes
    \item Tutoriel: création d'une commande avancée pour sauter jusqu'au prochain appel d'un service dans un composant Molecule
\end{enumerate}


\section{Outils avancés II : Seeker (30 minutes)}

\textit{Retour vers le futur, ou peut-être le passé, de toutes façons on peut faire les deux² !}

\begin{enumerate}
    \item Présentation et démonstration du \textit{Time-Traveling Debugger}
    \item Présentation et démonstration des \textit{Time-Traveling Queries}
\end{enumerate}


\section{Étendre le debugger avec des outils dédiés (1 heure)}

\begin{enumerate}
    \item Présentation du mécanisme d'extension (illustration, demo)
    \item Tutoriel: création d'une extension pour inspecter les interfaces des composants Molecule dans le debugger et pour y configurer des points d'arrêt
\end{enumerate}


\section{Temps d'échange et discussion}

\end{document}
